\documentclass[11pt]{amsart}
\usepackage{geometry}                % See geometry.pdf to learn the layout options. There are lots.
\geometry{a4paper}                   % ... or a4paper or a5paper or ... 
%\geometry{landscape}                % Activate for for rotated page geometry
%\usepackage[parfill]{parskip}    % Activate to begin paragraphs with an empty line rather than an indent

%\usepackage[spanish]{babel}
%\usepackage[applemac]{inputenc}

\usepackage{graphicx}
\usepackage{amssymb}
\usepackage{epstopdf}
\DeclareGraphicsRule{.tif}{png}{.png}{`convert #1 `dirname #1`/`basename #1 .tif`.png}

\title{Computaci\'on Gr\'afica}
\author{Frameworks de Programaci\'on v1.5}
%\date{}                                           % Activate to display a given date or no date

\begin{document}
\maketitle

\section{Introducci\'on}
El presente documento describe los diversos Frameworks implementados y distribuidos con el objetivo de simplificar la tarea de implementaci\'on de los conceptos estudiados a lo largo del curso de Computaci\'on Gr\'afica I de la Facultad de Ingenier\'ia y Tecnolog\'ias de la Universidad Cat\'olica del Uruguay.

Los Frameworks se componen en su mayor\'ia por una clase Canvas, la cual implementa la funcionalidad b\'asica para crear un Lienzo (rect\'angulo dibujable) sobre el cual se define la operación para pintar pixeles.

Las operaciones de pintado de Pixeles, denominadas \emph{putpixel}, pasan a trav\'es de un cambio de coordenadas que posiciona el punto (0,0) en el centro del Canvas.

Todos los Frameworks incluyen un archivo adicional que muestra c\'omo instanciar un nuevo Canvas, colocarlo dentro de la ventana (en caso de que sea necesario) y pintar un p\'ixel sobre \'el.

Los Frameworks para C y C++ incluyen, a su vez, abstracciones propias para representar colores. Tanto en estos Frameworks como en los de los dem\'as lenguajes, los valores de los canales rojo, verde y azul de cada color se espera que sean valores en el intervalo [0-255].

Todo el c\'odigo del cual se componen los Frameworks se encuentra disponible, por lo cual no duden en estudiarlo y familiarizarse con \'el, ya que deber\'an modificarlo a lo largo del curso para implementar los conceptos que se trabajen en clase.

Se dispone de versiones para los siguientes lenguajes: Java, C\#, Python, C++ y C. Una vez decidido el lenguaje, dirigirse directamente a la secci\'on correspondiente.

\section{Actualizaciones}

\begin{center}
	\begin{tabular}{ | c | p{12cm} | }
	\hline
	Versi\'on & Cambios \\ \hline
	1.5 & Reescritura de la documentaci\'on en \LaTeX. \\ & Mejoras a Frameworks C y C++.\\
	1.4 & Correci\'on de errores menores al Framework C++. \\
	1.3 & Mejoras de Performance al Framework C\#. \\
	1.2 & Mejoras de Performance al Framework C\#. \\
	1.1 & Mejoras de Performance al Framework C\#. \\
	1.0 & Versi\'on inicial. \\ \hline	
	\end{tabular}
\end{center}

\section{Framework Java}

\subsection{Descripci\'on}
La clase \emph{Canvas} se encuentra implementada en el archivo \emph{Canvas.java}. \'Esta se encuentra implementada como un \emph{JPanel} que debe ser agregado como "hijo" componente de un \emph{JFrame}.

Se incluye un ejemplo en el archivo \emph{Main.java}.

\subsection{Operaciones}
\begin{enumerate}
	\item \emph{public Canvas(int w, int h)} - Crear un nuevo \emph{Canvas} de dimensiones w x h.
	\item \emph{public void putpixel(int x, int y, Color c)} - Pintar un p\'ixel sobre el Canvas en (x,y) de color \emph{c}.
	\item \emph{public void clear()} - Borrar el contenido del \emph{Canvas}, reseteandolo a negro.
	\item \emph{public void repaint()} - Refrescar el \emph{Canvas}, mostrando los cambios realizados por las llamadas a putpixel. Es necesario llamarlo para mostrar nuevos pixels pintados.
\end{enumerate}

\section{Framework C\#}

\subsection{Descripci\'on}
La clase \emph{Canvas} se encuentra implementada en el archivo \emph{Canvas.cs}. Esta se encuentra implementada como un \emph{Panel} que debe ser agregado como "hijo" componente de un \emph{Frame}.

Se incluye un ejemplo en el archivo \emph{Form1.cs}.

\subsection{Operaciones}
\begin{enumerate}
	\item \emph{public Canvas(int w, int h)} - Crear un nuevo \emph{Canvas} de dimensiones w x h.
	\item \emph{public void putpixel(int x, int y, Color c)} - Pintar un p\'ixel sobre el Canvas en (x,y) de color \emph{c}.
	\item \emph{public void clear()} - Borrar el contenido del \emph{Canvas}, reseteandolo a negro.
	\item \emph{public void repaint()} - Refrescar el \emph{Canvas}, mostrando los cambios realizados por las llamadas a putpixel. Es necesario llamarlo para mostrar nuevos pixels pintados.
\end{enumerate}

\section{Framework Python}

\subsection{Descripci\'on}

La clase \emph{Canvas} se encuentra implementada en el archivo \emph{canvas.py}. \'Esta implementa la l\'ogica necesaria tanto para crear la ventana como para pintar p\'ixeles sobre la misma.

Se incluye un ejemplo en el archivo \emph{main.py}.

\subsection{Operaciones}
\begin{enumerate}
	\item \emph{public Canvas(int w, int h)} - Crear un nuevo \emph{Canvas} de dimensiones w x h.
	\item \emph{public void putpixel(int x, int y, Color c)} - Pintar un p\'ixel sobre el Canvas en (x,y) de color \emph{c}.
	\item \emph{public void clear()} - Borrar el contenido del \emph{Canvas}, reseteandolo a negro.
	\item \emph{public void repaint()} - Refrescar el \emph{Canvas}, mostrando los cambios realizados por las llamadas a putpixel. Es necesario llamarlo para mostrar nuevos pixels pintados.
\end{enumerate}

\section{Framework C++}

\section{Descripci\'on}
La clase \emph{Canvas} se encuentra implementada en los archivos \emph{canvas.h} y \emph{canvas.cpp}. \'Esta implementa la l\'ogica necasaria tanto para crear la ventana como para pintar p\'ixels sobre la misma.

Tambi\'en incluyen los archivos \emph{color.h} y \emph{color.cpp}, los cuales definen e implementan la clase \emph{Color} para representar colores, y un ejemplo en el archivo \emph{main.cpp}. A su vez, se incluye un archivo \emph{Makefile} que muestra c\'omo compliar el proyecto sobre un ambiente UNIX/Linux.

\subsection{Operaciones}
\begin{enumerate}
	\item \emph{public Canvas(int w, int h)} - Crear un nuevo \emph{Canvas} de dimensiones w x h.
	\item \emph{public void putpixel(int x, int y, Color c)} - Pintar un p\'ixel sobre el Canvas en (x,y) de color \emph{c}.
	\item \emph{public void clear()} - Borrar el contenido del \emph{Canvas}, reseteandolo a negro.
	\item \emph{public void repaint()} - Refrescar el \emph{Canvas}, mostrando los cambios realizados por las llamadas a putpixel. Es necesario llamarlo para mostrar nuevos pixels pintados.
\end{enumerate}

\section{Framework C}

\subsection{Descripci\'on}
Debido a que C es un lenguaje estructurado, no se dispone de una clase \emph{Canvas}, sino que el Framework se encuentra implementado en forma de funciones, dentro de los archivos \emph{canvas.h} y \emph{canvas.c}. Las funciones implementan la l\'ogica necesaria tanto para crear la ventana como para pintar p\'ixeles sobre la misma.

As\'i mismo, se incluye un tipo opaco \emph{Color} (definido en \emph{color.h}) por conveniencia y se incluye un archivo \emph{Makefile} que muestra c\'omo compliar el proyecto sobre un ambiente UNIX/Linux.

\subsection{Funciones}
\begin{enumerate}
	\item \emph{void cg\_init(int w, int h)} - Crear una ventana de dimensiones w x h.
	\item \emph{void cg\_putpixel(int x, int y, Color c)} - Pintar un p\'ixel sobre el Canvas en (x,y) de color \emph{c}.
	\item \emph{public cg\_clear()} - Borrar el contenido del \emph{Canvas}, reseteandolo a negro.
	\item \emph{void cg\_repaint(void)} - Refrescar el \emph{Canvas}, mostrando los cambios realizados por las llamadas a putpixel. Es necesario llamarlo para mostrar nuevos pixels pintados.
	\item \emph{void cg\_close(void)} - Liberar recursos.
\end{enumerate}


\end{document}